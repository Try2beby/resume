% !TEX TS-program = xelatex
% !TEX encoding = UTF-8 Unicode
% !Mode:: "TeX:UTF-8"

\documentclass{resume}
\usepackage{zh_CN-Adobefonts_external} % Simplified Chinese Support using external fonts (./fonts/zh_CN-Adobe/)
%\usepackage{zh_CN-Adobefonts_internal} % Simplified Chinese Support using system fonts
\usepackage{linespacing_fix} % disable extra space before next section
\usepackage{cite}

\begin{document}
\pagenumbering{gobble} % suppress displaying page number

\name{涂伟豪}

\basicInfo{
	\email{wy1171633763@163.com} \textperiodcentered\
	\phone{(+86) 17346651272} \textperiodcentered\
	\github[Try2beby]{https://github.com/Try2beby}}

\section{\faGraduationCap\  教育背景}
\mydatedsubsectionthree{\textbf{复旦大学}}{大数据学院}{2023年9月 -- 至今}
\textit{在读硕士研究生}\ 应用统计, 预计 2025 年 6 月毕业
\begin{itemize}
	\item \textbf{主修课程}: 知识图谱概念与技术,数据挖掘,神经网络和深度学习,社交网络分析等
\end{itemize}

\mydatedsubsectionthree{\textbf{北京航空航天大学}}{数学科学学院}{2019年9月 -- 2023年6月}
\textit{学士}\ 信息与计算科学
\begin{itemize}
	\item \textbf{主修课程}: 大学计算机基础(Python),数据结构;概率论,数理统计;最优化理论与算法等
	\item \textbf{学分绩点}: \textbf{3.81}/4.0 \quad \textbf{GPA专业排名}: 1/10
	\item \textbf{荣誉奖励}:
	      \begin{itemize}[label=$\ast$]
		      \item \textbf{国家奖学金}(2019-2020)、国家励志奖学金
		      \item 第四届美团商业分析精英大赛卓越作品奖,全国大学生数学建模竞赛北京赛区甲组一等奖,美国大学生数学建模竞赛H奖
		      \item 校学习优秀奖学金、校级优秀生
	      \end{itemize}
\end{itemize}

\section{\faUsers\ 项目经历}
\mydatedsubsectionthree{\textbf{RAG构建及优化}}{\textbf{AI Rudder} (实习)}{2024年2月--2024年5月}
\begin{itemize}
	\item 基于 LlamaIndex 完整搭建 RAG retrieval 流程. 尝试使用多种 embedding, sparse, 混合搜索, rerank, rank fusion模型(方法)优化 retrieval 指标;
	\item 基于 Llama 3 搭建流式对话系统,学习并实践一次性 prompt 编排.
\end{itemize}

\mydatedsubsectionthree{\textbf{第四届美团商业分析精英大赛}}{基于BERT的评论情感分析与评分预测}{2024年4月}
\begin{itemize}
	\item 使用python爬取上海市内餐饮店的评论数据,分18个细粒度使用GPT-3.5进行情感分类标注;
	\item 基于Bert-Base Chinese搭建 ACSA-RP 联合模型,使用带权的交叉熵损失处理类别不均衡问题;
	      % \item 在情感分类上准确率(非缺失类)达到70\%;评分回归任务中,误差呈现中心为0的单峰分布,MAE 为 0.4266. 均优于非基于BERT的模型;
	\item 负责模型搭建、训练;获\textbf{卓越作品奖} (49/792).
\end{itemize}


\mydatedsubsectionthree{\textbf{知识图谱构建}}{\textit{Deep Learning}书籍知识图谱构建}{2023年12月}
\begin{itemize}
	\item 利用GPT-3.5生成实体抽取数据集,微调BERT,将其用于全书实体的抽取;
	\item 以编辑距离和语义embedding为指标,辅以GPT4进行实体对齐;
	\item 使用BERT for MRE进行关系分类,构建知识图谱.
\end{itemize}

\mydatedsubsectionthree{\textbf{毕业设计}}{深度学习中基于对偶理论的训练算法探究}{2023年2月--2023年5月}
\begin{itemize}
	\item 实现了求解带耦合线性约束的Minimax问题的MGD算法,并将其用到神经网络的训练问题中;
	\item 开发用于求解神经网络训练问题的对偶问题的削平面算法中的问题结构,优化削平面算法的求解过程:
	      \begin{itemize}[label=$\ast$]
		      \item 简化了求解违反约束神经元子问题(SOCP)的求解;
		      \item 在解决约束数量递增的近似对偶问题(LP)时使用热启动,求解时间花费减少90\%.
	      \end{itemize}
\end{itemize}

% \section{\faCogs\ 能力素养}
% % increase linespacing [parsep=0.5ex]
% \begin{itemize}[parsep=0.5ex]
% 	\item \textbf{编程}\quad 熟练使用Python,掌握C语言、SQL,掌握数据结构知识;熟悉linux平台的使用;
% 	\item \textbf{机器学习}\quad
% 	      \begin{itemize}[parsep=0.5ex,label=$\ast$]
% 		      \item 学习深度学习、机器学习课程及阅读相关书籍,熟悉相关理论基础;
% 		      \item 熟悉神经网络搭建和训练,熟悉PyTorch的使用;
% 	      \end{itemize}
% \end{itemize}
\section{\faCogs\ 能力素养}
% increase linespacing [parsep=0.5ex]
\begin{itemize}[parsep=0.5ex]
	\item \textbf{编程}\quad 熟练使用Python,掌握C语言、SQL,掌握数据结构知识,了解Bash脚本控制程序自动化执行;熟悉linux平台的使用;
	\item \textbf{机器学习}
	      \begin{itemize}[parsep=0.5ex,label=$\ast$]
		      \item 学习Coursera平台上Andrew Ng主讲的Deep Learning系列课程并取得证书,有神经网络搭建和训练经验,熟悉PyTorch的使用;
		      \item 学习斯坦福cs229(Machine Learning)在线课程,掌握常见机器学习算法的理论基础和实现.
	      \end{itemize}
	\item \textbf{英语}\quad CET-4 590, CET-6 512\quad 有较强的英文读写能力;
\end{itemize}


\end{document}
